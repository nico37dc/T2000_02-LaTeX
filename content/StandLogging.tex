\section{Methoden des Loggings}

Logging kann durch verschiedene Ansätze und Techniken realisiert werden. Dieses Kapitel gibt
einen Überblick über die verschiedenen Logging-Ansätze, um die passende Lösung für die gegebenen Bedingungen zu finden.

Die Kategorisierung von Log-Nachrichten nach ihrer Priorität erfolgt durch verschiedene Log-Level. Typische Level sind \glqq Debug\grqq, \glqq Info\grqq, \glqq Warn\grqq, \glqq Error\grqq\ und \glqq Critical\grqq. 
Durch die Einstellung eines bestimmten Log-Levels kann gesteuert werden, welche Nachrichten in den Log-Dateien erscheinen. Dies erhöht die Übersichtlichkeit und 
verbessert die Performance, indem irrelevante Informationen ausgeblendet werden.

Die Struktur einer Log-Nachricht ist entscheidend für ihren Nutzen. Eine Log-Nachricht enthält Elemente wie den Zeitstempel, das Log-Level, die betroffene Komponente 
oder Funktion und eine beschreibende Nachricht. Die Anpassung der Formatierung ermöglicht die Integration spezifischer Anforderungen wie beispielsweise Benutzer-IDs. Dies 
erleichtert die Nachverfolgung und Diagnose von Ereignissen und Fehlern.

Um die Systemleistung nicht durch übermäßig große Log-Dateien zu beeinträchtigen, werden Verfahren wie Log-Rotation und Archivierung verwendet. Log-Rotation ist eine Methode, 
bei der die Protokollierung zwischen mehreren Dateien wechselt, sobald eine festgelegte Zeitdauer oder Dateigröße erreicht wird. Es ist möglich, Parameter wie die 
maximale Dateigröße oder die Zeitdauer festzulegen, nach denen eine Rotation erfolgen soll, sowie die Anzahl der zu rotierenden Log-Dateien zu definieren. Durch die Anwendung 
von zeit- oder größenbasierter Rotation werden ältere Log-Dateien entweder archiviert oder gelöscht, wodurch der Speicher effizient verwaltet und die 
Systemwartung vereinfacht wird.

Es wird zwischen synchronem und asynchronem Logging unterschieden. Synchrones Logging stellt die Reihenfolge und Vollständigkeit der Log-Einträge sicher, kann aber die 
Anwendungsleistung beeinträchtigen, da es die Log-Nachrichten innerhalb der Hauptanwendung erstellt. Asynchrones Logging hingegen verbessert die Performance, indem Log-Nachrichten 
in eine Warteschlange gestellt werden, die unabhängig von der Hauptanwendung verarbeitet wird.

Monitoring und Alarmierung sind entscheidend für die proaktive Überwachung und schnelle Reaktion auf potenzielle Probleme. Automatische Benachrichtigungen bei kritischen 
Log-Nachrichten unterstützen den Testingenieur dabei, den Testprozess effektiv zu überwachen.
 
Logs können auf Konsolen, in Dateien oder Datenbanken ausgegeben werden. Es ist dabei wichtig, sensible Daten zu schützen und Logs nur autorisierten Personen zugänglich zu machen. 
Dadurch wird sichergestellt, dass die Logging-Verfahren nicht nur effizient, sondern auch sicher sind und die Integrität und Vertraulichkeit der Systemdaten gewährleistet wird.

\citep{Beneken2022} \citep{Gu2023}