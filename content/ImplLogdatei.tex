\section{Umsetzung des Loggings} \label{Logging}

Um einen Logger mithilfe der Bibliothek \glqq spdlog\grqq\ zu implementieren, ist zunächst eine Initialisierung erforderlich. Diese erfolgt über eine Methode, die abhängig von 
den übergebenen Parametern eine Log-Rotation mit definierter maximaler Dateigröße und Anzahl an Dateien erstellt (siehe Anhang: Seite \pageref{initializeLogger} ab Zeile 395). 
Zusätzlich wird das Log-Level durch eine Hilfsfunktion festgelegt (siehe Anhang: Seite \pageref{setLogLevel} ab Zeile 356).

Nach der Initialisierung kann der Logger in Funktionen und Methoden eingebunden werden. Dies geschieht durch folgenden Aufruf:

\vspace{12pt}

\begin{lstlisting}[
    caption={Aufruf des Loggers},
    label={lst:logAufruf},
    language=C++,
    style=myCPP,
    basicstyle=\small\ttfamily,
    numberstyle=\small
]
auto loggerClient = spdlog::get("logClient");
\end{lstlisting}

Innerhalb der Callback-Funktion \glqq processRequest\grqq\ werden relevante Informationen zur Datenübertragung protokolliert. Es wird vermerkt, wenn ein Bildstapel empfangen wird, 
einschließlich der Anzahl der enthaltenen Bilder und der Größe jedes einzelnen Bildes. Auch nach erfolgreichem Speichern eines Bildes wird eine entsprechende Log-Nachricht erstellt. 
Zudem werden die Zeiten für die Erstellung und Speicherung der Bilder sowie deren Differenz dokumentiert. Treten Fehler auf, so werden diese ebenfalls sorgfältig festgehalten. Ein 
typisches Beispiel für einen solchen Fehler ist der Empfang eines Bildes außerhalb des vorgesehenen Containers. Dies deutet auf eine Fehlfunktion des RadarImagers hin, da unter 
regulären Bedingungen jedes Bild innerhalb eines Containers übermittelt wird.

Regelmäßig werden wichtige Statistiken wie \ac{FPS}, Frame Count und die Anzahl verlorener Bilder in die Log-Dateien geschrieben. Diese Daten werden über das Impact 
Acquire \acs{SDK} bezogen und auf Auffälligkeiten hin überprüft. Bei signifikanten Änderungen der \ac{FPS} oder einer Zunahme verlorener Bilder im Vergleich zur letzten Überprüfung 
wird eine Warnung ausgegeben, um im Nachinein mögliche Probleme identifizieren zu können. Dies ist besonders relevant, wenn ein Fehler zum Abbruch des Tests führt, beispielsweise 
bei einer fehlerhaften Anfrage.

Zum Monitoring wird die \glqq cURL\grqq-Bibliothek genutzt, um über eine HTTP-Anfrage ein \glqq Azure Issue\grqq\ zu erstellen. Die Methode \glqq createAzureIssue\grqq\
in der Klasse \glqq AzureIssueCreator\grqq\ sendet eine HTTP POST-Anfrage an die Azure DevOps REST-API, um ein \glqq Azure Issue\grqq\ mit spezifischen Feldern wie Titel, 
Beschreibung und zugewiesenem Benutzer zu erstellen (siehe Anhang: Seite \pageref{AzureIssueCreator1} ab Zeile 146 und \pageref{AzureIssueCreator2} ab Zeile 502). Die Authentifizierung erfolgt über ein in \glqq Base64\grqq\ kodiertes \ac{PAT}. Die Serverantwort wird protokolliert und 
relevante Informationen werden in die Log-Dateien geschrieben.

Um das Testprogramm automatisch zu beenden, falls über einen längeren Zeitraum keine Bilder empfangen werden, überwacht eine while-Schleife im Hauptprogramm die Zeit seit der letzten 
Anfrage. Diese Schleife endet entweder durch Überschreitung dieser Zeitgrenze oder durch das Drücken einer Taste (siehe Anhang: Seite \pageref{while-Schleife} ab Zeile 155). Eine Hilfsfunktion ermöglicht dabei die parallele Abfrage der 
Tastatureingabe.

Zusätzlich zum Logging der Datenübertragung und Speicherung wird die separate Log-Datei zur Überwachung des korrekten Ablaufs des Testprogramms geführt. Diese Log-Datei wird unabhängig 
von den Hilfsfunktionen für das Logging verwaltet, um auch dort mögliche Fehlerquellen zu identifizieren.