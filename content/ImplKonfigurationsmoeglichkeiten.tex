\section{Umsetzung der Konfigurationsmöglichkeiten}

Die Implementierung der verschiedenen Konfigurationsmöglichkeiten für die Speicherung von Bildern oder Bildstapeln wird in der Callback-Funktion \glqq processRequest\grqq\
sowie in den zugehörigen Hilfsfunktionen der Klasse \glqq ImageSavePreparer\grqq\ (siehe Anhang: Seite \pageref{processRequest} ab Zeile 6, \pageref{ImageSavePreparer1} ab Zeile 106 und \pageref{ImageSavePreparer2} ab Zeile 246) realisiert. Bei Empfang eines Bildes oder Bildstapels prüft die Funktion zunächst, ob eine 
Speicherung gemäß der aktuellen Konfigurationseinstellungen erforderlich ist.

Anschließend wird unterschieden, ob es sich um ein einzelnes Bild oder um einen Bildstapel handelt. Diese Unterscheidung basiert auf der Anzahl der Bilder (Parts), die der 
empfangene Container enthält. Der Speichervorgang variiert je nachdem, ob ein einzelnes Bild oder ein Bildstapel vorliegt. Listing \ref{lst:save} zeigt diese Differenzierung 
und den Ablauf des Speichervorgangs:

\vspace{12pt}

\begin{lstlisting}[
    caption={Speichervorgang},
    label={lst:save},
    language=C++,
    style=myCPP
]
@\textcolor{CommentGreen}{// Save images depending on the configuration}@
@\textcolor{KeywordPurple}{if}@ (@\textcolor{black}{param}@.@\textcolor{black}{saveImages}@ && (@\textcolor{black}{param}@.@\textcolor{black}{saveImagesInterval}@ == 1 || @\textcolor{black}{pRequest}@->@\textcolor{black}{infoFrameID}@.@\textcolor{FunctionsYellow}{read}@() % @\textcolor{black}{param}@.@\textcolor{black}{saveImagesInterval}@ == 1))
{
  @\textcolor{KeywordPurple}{if}@ (@\textcolor{black}{bufferPartCount}@ == 1)  @\textcolor{CommentGreen}{// Buffer has exactly one image}@
  {
    @\textcolor{TypeBlue}{int}@ @\textcolor{black}{imageNumber}@ = @\textcolor{FunctionsYellow}{ceil}@((@\textcolor{TypeBlue}{float}@)@\textcolor{black}{pRequest}@->@\textcolor{black}{infoFrameID}@.@\textcolor{FunctionsYellow}{read}@()/(@\textcolor{TypeBlue}{float}@)@\textcolor{black}{param}@.@\textcolor{black}{saveImagesInterval}@);
    @\textcolor{TypeBlue}{string}@ @\textcolor{black}{filename}@ = @\textcolor{black}{save}@.@\textcolor{FunctionsYellow}{prepareImageSave}@(@\textcolor{black}{param}@, @\textcolor{black}{imageNumber}@);
    @\textcolor{black}{pRequest}@->@\textcolor{FunctionsYellow}{getBufferPart}@(0).@\textcolor{FunctionsYellow}{getImageBufferDesc}@().@\textcolor{FunctionsYellow}{save}@(@\textcolor{black}{filename}@);
    @\textcolor{black}{loggerClient}@->@\textcolor{FunctionsYellow}{info}@("@\textcolor{StringOrange}{Saved image \{\} to \{\}}@", @\textcolor{black}{pRequest}@->@\textcolor{black}{infoFrameID}@.@\textcolor{FunctionsYellow}{read}@(), @\textcolor{black}{filename}@);
  }
  @\textcolor{KeywordPurple}{else}@  @\textcolor{CommentGreen}{// Buffer has more than one image}@
  {
    @\textcolor{TypeBlue}{string}@ @\textcolor{black}{filename}@ = @\textcolor{black}{save}@.@\textcolor{FunctionsYellow}{prepareBufferSave}@(@\textcolor{black}{param}@, @\textcolor{black}{pRequest}@->@\textcolor{black}{infoFrameID}@.@\textcolor{FunctionsYellow}{read}@(), @\textcolor{black}{i}@);
    @\textcolor{black}{pRequest}@->@\textcolor{FunctionsYellow}{getBufferPart}@(@\textcolor{black}{i}@).@\textcolor{FunctionsYellow}{getImageBufferDesc}@.@\textcolor{FunctionsYellow}{save}@(@\textcolor{black}{filename}@);
    @\textcolor{black}{loggerClient}@->@\textcolor{FunctionsYellow}{info}@("@\textcolor{StringOrange}{Saved image \{\} part \{\} to \{\}}@", @\textcolor{black}{pRequest}@->@\textcolor{black}{infoFrameID}@.@\textcolor{FunctionsYellow}{read}@(), @\textcolor{black}{i}@, @\textcolor{black}{filename}@);
  }
}
\end{lstlisting}

\paragraph{Einzelbild}

Um die Nummer des aktuell zu speichernden Bildes zu bestimmen, wird die Identifikationsnummer des Bildes durch das festgelegte Speicherintervall geteilt. Diese Nummer 
wird für die Vorbereitung des Speichervorgangs verwendet. In diesem Schritt wird der Pfad zum Speicherort generiert und bei Bedarf ein neuer
Ordner angelegt. Hierbei wird zwischen der Nummerierung der Ordner und der Benennung mittels eines Zeitstempels unterschieden.

Bei Nummerierung setzt sich der Pfad aus dem Hauptverzeichnis für Bilder und einem Unterverzeichnis, das die aktuelle Ordner-Nummer trägt, zusammen. Erreicht das Speicherintervall 
den Punkt, an dem ein neuer Ordner benötigt wird, wird dieser mit der entsprechenden Nummer erstellt. Im Falle der Benennung durch einen Zeitstempel erhalten die Unterverzeichnisse 
Namen, die den Zeitstempel bis auf Millisekunden genau wiedergeben. Die Benennung auf die Millisekunde genau ist notwendig, um das unabsichtliche Überschreiben der Verzeichnisse zu verhindern, da je nach Konfiguration 
mehrere Verzeichnisse pro Sekunde erstellt werden. Es ist dabei zusätzlich erforderlich, den Namen des aktuellen oder vorherigen Ordners zu speichern, da der Zeitstempel nicht 
einfach rekonstruiert werden kann für die nächsten Bilder, die im selben Ordner abgelegt werden sollen.

Nachdem Erstellen des Pfads und gegebenenfalls dem Anlegen eines neuen Ordners, übergibt die Untermethode \glqq createImageSubpathFolder\grqq\ den Pfad an die Methode \glqq prepareImageSave\grqq\ (siehe Anhang: Seite \pageref{ImageSavePreparer2} ab Zeile 246). 
Diese ist verantwortlich für die Erstellung des Bildnamens und somit für die Festlegung des endgültigen Speicherorts des Bildes. Auch hier wird zwischen Nummerierung und Zeitstempel 
differenziert.

Schließlich wird das Bild am festgelegten Speicherort durch den Aufruf der \glqq save\grqq-Methode des Impact Acquire SDK gespeichert.

\paragraph{Bildstapel}

Bei der Verarbeitung eines Bildstapels erfolgt die Vorbereitung ähnlich wie bei Einzelbildern, allerdings wird für jeden Container ein neuer Ordner angelegt. Innerhalb einer 
for-Schleife, in der auch der Code aus Listing \ref{lst:save} implementiert ist, wird die Hilfsvariable \glqq i\grqq\ inkrementiert, die das aktuelle Bild im Stapel kennzeichnet. 
Ein neuer Ordner wird immer dann erstellt, wenn \glqq i\grqq\ den Wert 0 annimmt, der das erste Bild eines neuen Bildstapels signalisiert.

Die Methode \glqq prepareBufferSave\grqq\ ist anschließend dafür zuständig, den Namen der Bilddatei zu bestimmen (siehe Anhang: Seite \pageref{prepareBufferSave} ab Zeile 335). Abhängig von der Konfiguration wird hierbei entweder die Nummer des 
Bildes innerhalb des Stapels (\glqq i\grqq) oder ein Zeitstempel verwendet. In jedem Fall wird die Nummer des Containers in die Benennung der Datei integriert.

Das aktuelle Bild wird dann am vorbereiteten Speicherort abgelegt. Dieser Prozess wiederholt sich durch die for-Schleife für jedes Bild des Stapels.