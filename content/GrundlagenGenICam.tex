\section{Grundlagen des GenICam-Standards}

GenICam, kurz für \glqq Generic Interface for Cameras\grqq, ist ein Standard von der \ac{EMVA}, der eine einheitliche 
Programmierschnittstelle für verschiedene Arten von Kameras in der industriellen Bildverarbeitung definiert. Der Standard umfasst 
mehrere Module, darunter GenApi, GenTL, GenCP und GenDC, welche verschiedene Aspekte der Kameraschnittstellen und -kommunikation abdecken.

Das \textbf{GenApi} (Generic Application Programming Interface) definiert eine XML-basierte Beschreibungssprache, die die Funktionen und Parameter 
einer Kamera beschreibt. Durch diese Beschreibung können Anwendungen die Kameraeinstellungen dynamisch auslesen und konfigurieren, ohne 
auf herstellerspezifische Treiber angewiesen zu sein.

Der \textbf{GenTL} (Generic Transport Layer) legt eine standardisierte Schnittstelle für die Datenübertragung zwischen der Kamera und dem Host-System fest. 
Dies ermöglicht die Entwicklung von Anwendungen, die unabhängig vom verwendeten Übertragungsprotokoll (z.B. GigE Vision, Camera Link) sind. 
Ein GenTL Producer stellt die erforderlichen Funktionen bereit, um Datenströme von der Kamera zu empfangen und zu verwalten.

Das \textbf{GenCP} (Generic Control Protocol) definiert ein standardisiertes Protokoll für die Steuerung und Konfiguration von Kameras. Es erlaubt die 
Kommunikation zwischen der Kamera und der Anwendung, sodass Konfigurationsparameter gesetzt oder abgefragt sowie Befehle an die Kamera gesendet werden können.

Der \textbf{GenDC} (Generic Data Container) ist ein Format für die standardisierte Beschreibung und Übertragung von Bild- und Metadaten. Die Speicherung komplexer Datenstrukturen, 
wie beispielsweise Bildstapel (3D-Bilddaten), erfolgt in einem einheitlichen Containerformat. Dies erlaubt eine erleichterte Interpretation und Verarbeitung der Daten durch 
unterschiedliche Anwendungen und Systeme.

Die vom RadarImager-System erfassten Daten werden in Form eines Bildstapels verarbeitet. Diese Bildstapel setzen sich aus mehreren einzelnen Bildlagen zusammen, 
welche in einem GenICam GenDC bereitgestellt werden. Der Einsatz des GenDC gewährleistet eine einheitliche und standardisierte Darstellung der Bilddaten, wodurch eine 
erleichterte Weiterverarbeitung und Analyse möglich ist. Die Übertragung der Bilddaten erfolgt über ein GenICam GenTL Producer Interface, welches die Daten vom Kamerasystem 
über Gigabit-Ethernet an das Host-System sendet.

\citep{BalluffRIWebsite} \citep{EMVA2024}