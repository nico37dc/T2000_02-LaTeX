\chapter{Theoretische Grundlagen}

\section{erste Beispiele}

\paragraph{Beispiel A}

Beispiel Aufzählung:
\begin{itemize}
    \item \textbf{Testplanung:} Diese Phase umfasst die Erstellung des Testkonzepts und des detaillierten Testplans. Hierbei werden das Testobjekt definiert, die erforderliche Testumgebung beschrieben, die Konfiguration des Testsystems festgelegt und die benötigten Testressourcen bestimmt. Die Testplanung legt den Grundstein für alle nachfolgenden Testaktivitäten und definiert den Umfang sowie die Werkzeuge, die für die Tests verwendet werden sollen.
    \item \textbf{Testdesign:} In dieser Phase werden die Testanforderungen verfeinert und spezifiziert. Es werden Testszenarien entwickelt und Kriterien für den Abschluss der Tests festgelegt. Je nach Umfang und Komplexität des Projekts kann diese Phase eng mit der Testplanung verbunden sein.
\end{itemize}

\paragraph{Beispiel B}

Beispiel Quelle: \citep{Witte2019} \citep{Witte2023}