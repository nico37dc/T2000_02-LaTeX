\section{Konzeptionelle Gestaltung der Konfiguration}

Die Konfiguration der Tests wird mithilfe einer Konfigurationsdatei vorgenommen. Dies bietet den Vorteil der Flexibilität sowie der Trennung von Quellcode 
und Konfiguration. Darüber hinaus kann die Konfigurationsdatei wiederverwendet werden, wobei bei Bedarf nur einzelne Parameter angepasst werden müssen. Zudem ist die Konfiguration 
stets nachvollziehbar, wodurch die Datei bereits eine Art Dokumentation des Testdurchlaufs ist. Ein weiterer Vorteil besteht darin, dass eine Erweiterung, beispielsweise durch 
eine \acs{GUI}, möglich ist. Es ist ratsam, dies klar zu trennen, sodass die \acs{GUI} die Konfigurationsdatei beschreibt, um die Lösung ohne \acs{GUI} selbst funktionsfähig zu halten.

Auf die ausschließliche Nutzung einer \acs{GUI} wird verzichtet, da diese für notwendige kurzfristige Anpassungen zu unflexibel ist. Zudem bleibt damit die Möglichkeit bestehen, das 
Testprogramm in ein Softwaretool zur weiteren Automatisierung zu integrieren. Terminaleingaben werden nicht genutzt, da sie für die wichtige Nachverfolgbarkeit und Konsistenz 
ungeeignet sind und keine einfache Handhabung bieten. Auch die Konfiguration der Testparameter direkt im Quellcode wird vermieden, da eine Neukompilierung für jeden 
Testdurchlauf nicht praktikabel ist. Dennoch werden innerhalb des Quellcodes bestimmte Einstellungen vorgenommen oder Konstanten definiert, die für das Testprogramm relevant sind. 
Diese betreffen jedoch allgemeine Aspekte, die nicht direkt mit spezifischen Testdurchläufen verbunden sind, wie beispielsweise die Festlegung von Grenzwerten zur Überprüfung der 
Plausibilität bestimmter Parameter.

Als Dateiformat für die Konfigurationsdatei wird JSON verwendet, da dies sowohl für den Benutzer als auch in C++ gut lesbar und verständlich ist. Zudem ermöglicht JSON eine 
strukturierte Darstellung der einzelnen Parameter. Dies verschafft dem Benutzer einen besseren Überblick und vereinfacht das Eintragen der Parameter. Innerhalb des Projekts 
existiert bereits eine Konfigurationsdatei, die allgemeine Parameter des Tests sowie Einstellungen für den simulierten Trigger enthält. Diese wird nach entsprechender Anpassung 
weiterverwendet. In Listing \ref{lst:config} ist die angepasste Konfigurationsdatei dargestellt:

\vspace{12pt}

\lstinputlisting[
    caption={Konfigurationsdatei},
    label={lst:config},
    language=json,
    style=myJSON
]{code/configParameter.json}

Um die verschiedenen Bereiche des Tests klar zu trennen, werden in der Konfigurationsdatei unterschiedliche Abschnitte verwendet, in denen jeweils die Parameter festgelegt werden. 
Somit ist die Konfigurationsdatei in drei Teile gegliedert. Der Abschnitt \glqq Basic\grqq\ enthält grundlegende Informationen über den Testdurchlauf, wie eine eindeutige Test-ID, eine 
Beschreibung und die Dauer des Testdurchlaufs. Im Abschnitt \glqq GenICam-Client\grqq\ sind Parameter für die Speicherung der Bilder und das Erstellen der Log-Dateien hinterlegt. Im letzten 
Teil \glqq objectParameter\grqq\ sind Parameter enthalten, die sich auf das zu testende Objekt bzw. den simulierten Trigger beziehen.

Um Missverständnisse zu vermeiden und die korrekte Konfiguration der Testdurchläufe zu gewährleisten, existiert zu der Konfigurationsdatei eine Dokumentation. Diese beschreibt 
den Aufbau der Datei, erklärt die einzelnen Parameter und gibt deren Einheit sowie mögliche Werte an.

Damit die Parameter aus der Konfigurationsdatei korrekt ausgelesen und weiterverwendet werden, wird in C++ die Bibliothek \glqq nlohmann/json\grqq\ verwendet. Diese ermöglicht die 
Überprüfung der Datentypen der einzelnen Eingaben und die Speicherung der Parameter. Die Bibliothek wird verwendet, da sie benutzerfreundlich ist, konform mit allen JSON-Datentypen 
arbeitet und zudem eine umfassende Dokumentation besitzt.