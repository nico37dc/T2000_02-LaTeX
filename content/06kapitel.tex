\chapter{Bewertung und Fazit}

In dieser Projektarbeit wird ein zuverlässiges Konzept für die Konfiguration und die automatisierte Generierung von Log-Dateien zum Testen des RadarImagers entwickelt. Das Konzept 
wird in C++ unter Einsatz verschiedener Bibliotheken sowie des Impact Acquire SDKs implementiert. Es ermöglicht, die Testparameter in einer Konfigurationsdatei festzulegen, den Test 
automatisiert durchzuführen und Auffälligkeiten in Log-Dateien zu dokumentieren. Dies erleichtert es dem Testingenieur, Fehler und andere Unregelmäßigkeiten des RadarImagers zu 
identifizieren.

Das entwickelte Testprogramm bietet eine einfache Konfiguration und eine zuverlässige Dokumentation der Ereignisse in Log-Dateien. Es stellt jedoch keine endgültige Lösung dar, 
da jedes Testprogramm im agilen Umfeld kontinuierlicher Weiterentwicklung und Anpassung an sich verändernde Bedingungen bedarf. Das aktuelle Konzept und Programm bieten eine sowohl konsistente als 
auch flexible Basis für zukünftige Anpassungen. Durch das Programm ist der Testablauf nun automatisierter und flexibler geworden, obwohl eine vollständige Automatisierung noch nicht 
erreicht ist. Die zu Beginn des Projekts gestellten Ziele werden jedoch erfolgreich erreicht.

Für die Zukunft besteht das Potenzial, das Programm weiterzuentwickeln oder zu erweitern, um den Grad der Automatisierung zu erhöhen. Beispielsweise könnte die Konfiguration durch 
die Entwicklung einer benutzerfreundlichen \acs{GUI} intuitiver gestaltet werden. Zudem wäre die Integration des Testprogramms in ein übergeordnetes Softwaretool, beispielsweise in 
einen Azure DevOps Test Plan sinnvoll, um den Test weiter zu automatisieren und in bestehende Verfahren einzubinden.

Das entwickelte Konzept lässt sich auch auf andere Projekte übertragen. Obwohl jedes Projekt individuelle Anforderungen hat, sind bestimmte Ansätze wie die Verwendung und Struktur 
einer Konfigurationsdatei oder die Rotation der Log-Dateien oft universell anwendbar.

Ein zuverlässiges Testkonzept und -programm sowie die Automatisierung von Tests sind jedoch allein nicht ausreichend für einen erfolgreichen Testprozess. Es ist ebenso entscheidend, 
sinnvolle Testfälle zu erstellen, die das System umfassend beanspruchen und so dazu beitragen, mögliche Fehler effektiv aufzudecken.