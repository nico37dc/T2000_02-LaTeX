\section{Definition der Anforderungen}

Es ist essentiell, die Anforderungen an das Testprogramm und die Testdurchführung präzise zu definieren. Zu diesem Zweck finden Gespräche mit Entwicklern und dem Testingenieur
statt, um verschiedene Ansätze zu erörtern und die spezifischen Anforderungen festzulegen. Besonderes Augenmerk liegt dabei auf den Parametern, die für jeden Testdurchlauf flexibel 
gewählt werden. Dieses Kapitel behandelt sowohl die grundlegenden Anforderungen als auch die erforderlichen Konfigurationsmöglichkeiten und das Logging.

Die grundlegenden Anforderungen sind entscheidend für die Funktionalität und die Interaktion mit dem Benutzer. Dazu zählt vor allem die Bedienbarkeit des Testprogramms. 
Es muss gewährleistet sein, dass das Testprogramm einfach und intuitiv zu bedienen ist und dass die Parameter schnell und verständlich eingestellt werden können. Dafür ist auch eine umfassende 
Dokumentation dieser Parameter erforderlich. Ein weiterer kritischer Aspekt ist die Zuverlässigkeit des Testprogramms. Es muss sichergestellt werden, dass der Test 
reproduzierbar ist und dass das Testprogramm keine zusätzlichen Fehler verursacht. Eventuell auftretende Fehler müssen klar identifiziert und adressiert werden. Um 
Fehler während der Testdurchführung zu minimieren, ist eine umfassende Validierung des Testprogramms notwendig, einschließlich der Implementierung von Modultests.

Darüber hinaus muss der Quellcode des Testprogramms nachvollziehbar gestaltet sein, um den Anwendern ein tiefgehendes Verständnis der Funktionsweise zu ermöglichen und Anpassungen 
an veränderte Anforderungen zu erleichtern. Die Kompatibilität des Testprogramms mit der Testumgebung ist ebenfalls von großer Bedeutung, ebenso wie die Möglichkeit, das Testprogramm 
zu erweitern oder in übergeordnete Softwarelösungen zu integrieren.

In Bezug auf die Konfiguration des Testprogramms soll der Benutzer die Möglichkeit haben, die Speicherung der vom RadarImager gesendeten Bilder zu steuern. 
Der Benutzer soll ein Intervall zur Speicherung der Bilder festlegen können, sodass beispielsweise nur jedes x-te Bild gespeichert wird. Zudem soll sich einrichten lassen,
nach wie vielen Einzelbildern ein neuer Ordner erstellt wird. Jeder Bilderstapel wird in einem eigenen Ordner gespeichert, um eine klare Trennung zu gewährleisten. 
Die Benennung dieser Ordner und der einzelnen Bilder soll ebenfalls konfigurierbar sein, wobei der Benutzer zwischen Nummerierung und Zeitstempeln wählen kann.

Das Logging soll ein- und ausschaltbar sein. Zudem sollen verschiedene Log-Level zur Verfügung stehen, die der Benutzer auswählen kann. Sollte ein weiterführendes 
Logging-Konzept wie Log-Rotation implementiert werden, müssen auch hierfür spezifische Parameter einstellbar sein. Das Ziel des Loggings ist es, dem Testingenieur umfassende 
Informationen über den Testdurchlauf zu liefern, während gleichzeitig die Übersichtlichkeit der Log-Dateien gewahrt bleibt.

Als Ausgabe soll das Testprogramm einen spezifisch benannten Ausgabeordner erstellen, der in Unterverzeichnissen die Bilder sowie die Log-Dateien des Testdurchlaufs enthält. 
Der Speicherort dieses Ordners soll konfigurierbar sein.

Zur Realisierung dieser Anforderungen ist es sinnvoll, im nächsten Schritt detaillierte Konzepte für die Konfiguration und das Logging zu entwickeln, wobei im agilen Umfeld eine 
gewisse Flexibilität in der Umsetzung berücksichtigt werden muss.