\section{Grundlagen von C++}

C++ ist eine weit verbreitete und leistungsstarke Programmiersprache, die in verschiedenen Anwendungsbereichen der Softwareentwicklung eingesetzt wird. 
Die Programmiersprache ist eine Erweiterung von C und bietet zusätzliche Funktionen und Möglichkeiten, die über die von C hinausgehen.
Ein wesentliches Merkmal von C++ ist die Fähigkeit, sowohl prozedurale als auch objektorientierte Programmierung zu unterstützen. Dies erlaubt es Entwicklern
sowohl strukturierten Code, der auf Funktionen und Prozeduren basiert, als auch objektorientierten Code, der auf Klassen und Objekten basiert, zu schreiben. 
Die Flexibilität der Sprache ermöglicht es, den Code gemäß den jeweiligen Anforderungen und Präferenzen zu strukturieren.

Des Weiteren zeichnet sich C++ durch seine Effizienz und die Kontrolle über die Hardware aus. Die Sprache ermöglicht es Entwicklern, direkt auf den Speicher und 
die Ressourcen eines Systems zuzugreifen. Dies ist insbesondere in Anwendungsbereichen wie in eingebetteten Systemen von 
Vorteil, da dort eine hohe Leistung sowie eine effiziente Nutzung der Ressourcen erforderlich sind.
Zur Unterstützung der Softwareentwicklung bietet C++ eine Vielzahl von Funktionen und Bibliotheken. Zu den wichtigsten Bibliotheken gehört
die \ac{STL}, die eine Vielzahl von Datenstrukturen und Algorithmen bereitstellt. Die Verwendung dieser Bibliotheken erlaubt es Entwicklern auf 
bewährte Lösungen zurückzugreifen und Zeit bei der Implementierung grundlegender Funktionen einzusparen.

Die Syntax von C++ ähnelt der von C, erweitert diese jedoch um zusätzliche Konstrukte und Schlüsselwörter, die speziell für die objektorientierte Programmierung 
konzipiert sind. Darüber hinaus unterstützt C++ fortgeschrittene Konzepte wie Vererbung, Polymorphismus, Templates und Ausnahmebehandlung, wodurch die Wiederverwendbarkeit 
und Flexibilität des Codes verbessert werden. Die genannten Konzepte ermöglichen es Entwicklern, komplexe Softwarearchitekturen zu entwerfen und den Code besser zu organisieren.
C++ findet Anwendung in einer Vielzahl von Bereichen, darunter Desktop-Anwendungen, Spieleentwicklung, eingebettete Systeme, Datenbanken und künstliche Intelligenz. 
Diese Vielseitigkeit macht C++ zu einer wichtigen Programmiersprache, welche die Entwicklung von leistungsstarken und effizienten Softwarelösungen erlaubt.

\citep{Stroustrup2015}