\chapter{Einführung}

\section{Problemstellung}

Die Entwicklung eines industriellen 3D-Bildgebungssytems auf Basis von Radartechnologie (RadarImager) erfordert umfangreiche Tests zur Gewährleistung 
der korrekten Funktionalität und Datenübertragung. Hierfür existiert bereits ein provisorisches Testprogramm, welches jedoch nicht dem gewünschten Funktionsumfang entspricht.
Es soll eine einfache Parametrisierung und Konfiguration des Tests möglich sein, um das Einstellen von verschiedenen Testbedingungen zu ermöglichen. Zudem müssen das Verhalten 
sowie mögliche Fehler des RadarImagers beziehungsweise der Datenübertragung aufgezeichnet werden und nachvollziehbar sein. Das Testkonzept sowie das dazugehörige Programm 
dürfen keine weiteren Fehler und somit Unsicherheiten erzeugen.

\section{Zielsetzung}

Ziel dieser Arbeit ist die Entwicklung eines Konzepts für die Konfiguration und Parametrisierung von Tests mittels einer Konfigurationsdatei und
die automatisierte Generierung von Log-Dateien. Dieses Konzept soll in C++ implementiert und gründlich auf seine Funktionalität getestet werden. Am
Ende des Projekts sollen die Testparameter in der Konfigurationsdatei eingestellt werden, sodass der Test automatisiert durchgeführt und mögliche
Fehler über die Log-Dateien identifiziert werden können. Dadurch sollen Fehler und Auffälligkeiten des RadarImagers beziehungsweise in der Datenübertragung erkannt werden.

\section{Vorgehensweise}

Die Arbeit beginnt mit einer gründlichen Einarbeitung in die Thematik, einschließlich der GenICam-Technologie, der C++-Programmiersprache und
des Verhaltens des RadarImagers. Die provisorische Version des Testprogramms wird dabei ebenfalls berücksichtigt. Die Anforderungen
werden mit dem zuständigen Testingenieur abgestimmt. Anschließend wird das Testkonzept unter Berücksichtigung der gesammelten Anforderungen an
den Test ausgearbeitet und in C++ implementiert. Zusätzlich wird der Aufbau der Konfigurationsdatei und der Log-Dateien festgelegt. Dabei steht die Automatisierung des
Testvorgangs im Fokus. Um die Funktionalität sicherzustellen und Fehler durch das Testprogramm zu vermeiden, werden nach jeder Erweiterung des
Funktionsumfangs weitere Modultests implementiert.