\section{Verarbeitung der Konfigurationsdatei} \label{Konfigurationsdatei}

Um die Parameter aus der Konfigurationsdatei effektiv im Code zu nutzen, empfiehlt es sich, diese zunächst in Variablen zu speichern. Hierfür wird eine Struktur angelegt, die 
sowohl die vom Benutzer festgelegten Werte als auch daraus resultierende Werte enthält. Bevor die Parameter aus der JSON-Datei in die Variablen übernommen werden, ist eine 
Überprüfung der Datentypen notwendig. Dies verhindert unerwünschte implizite Datentypkonvertierungen und Fehler. Die Verarbeitung der JSON-Datei erfolgt mithilfe der 
\glqq nlohmann/json\grqq-Bibliothek, die eine robuste und effiziente Handhabung von JSON-Daten ermöglicht.

\vspace{6pt}

\begin{lstlisting}[
    caption={Parsen der JSON-Datei},
    label={lst:JSON-Parse},
    language=C++,
    style=myCPP,
    basicstyle=\small\ttfamily,
    numberstyle=\small
]
ifstream paramFile(configFilename);
nlohmann::json root;
paramFile >> root;
\end{lstlisting}

Zunächst wird ein Input-File-Stream-Objekt erstellt, um die Konfigurationsdatei zu öffnen. Anschließend wird eine Variable \glqq root\grqq\ definiert, die als Container für 
die JSON-Daten dient. Der Inhalt von \glqq paramFile\grqq\ wird in die Variable \glqq root\grqq\ geparsed. Dadurch wird der Inhalt der JSON-Datei in eine strukturierte Form 
umgewandelt, die programmatisch einfacher zu verarbeiten ist. Dann kann der Datentyp eines Parameters überprüft werden:

\vspace{12pt}

\begin{lstlisting}[
    caption={Überprüfung des Datentyps},
    label={lst:Datentyp-Test},
    language=C++,
    style=myCPP,
    basicstyle=\small\ttfamily,
    numberstyle=\small
]
root["Basic"]["testID"].is_string()
\end{lstlisting}

Mithilfe der eckigen Klammern wird durch die Struktur der JSON-Datei navigiert. Die Methode \glqq is\_string()\grqq\ überprüft, ob der Wert an der angegebenen Stelle vom Typ
\glqq string\grqq\ ist. Falls einer der Datentypen inkorrekt ist, wird eine Fehlermeldung ausgegeben und das Testprogramm abgebrochen. Sind alle Datentypen korrekt, werden 
die Parameter in die vorgesehene Struktur übertragen. Hierbei ist \glqq param\grqq\ eine Instanz der Struktur für die Parameter (siehe Anhang: Seite \pageref{ParamStruct} ab Zeile 32).

\vspace{12pt}

\begin{lstlisting}[
    caption={Übernahme der Parameter},
    label={lst:Parameter-Übernahme},
    language=C++,
    style=myCPP,
    basicstyle=\small\ttfamily,
    numberstyle=\small
]
param.testID = root["Basic"]["testID"];
\end{lstlisting}

Anschließend erfolgt eine Plausibilitätsprüfung der Parameter. Dabei wird beispielsweise überprüft, dass eingegebene Strings nicht leer sind oder Pfade nur zulässige Zeichen 
enthalten. Bei numerischen Werten wird kontrolliert, ob diese im zugelassenen Bereich liegen, wie etwa beim Log-Level, das nur vier Werte (0 bis 3) zulässt.

Nach erfolgreicher Überprüfung werden der Ausgabeordner und die zugehörigen Unterverzeichnisse erstellt, die zur besseren Nachverfolgbarkeit einen Zeitstempel im Namen enthalten. Außerdem
wird dadurch vermieden, dass einer der Ordner unabsichtlich überschrieben wird.
Die Klasse \glqq TimestampProvider\grqq\ generiert diesen Zeitstempel, wobei zwischen sekunden- und millisekundengenauer Genauigkeit gewählt werden kann (siehe Anhang: Seite \pageref{TimestampProvider1} ab Zeile 66 und \pageref{TimestampProvider2} ab Zeile 5). Bei Problemen während des 
Erstellens der Ordner wird eine Fehlermeldung ausgegeben und in die Log-Datei für das Testprogramm geschrieben.

Die Methoden und Konstanten für die Parameterbereiche sind in der Klasse \glqq ConfigurationHandler\grqq\ abgelegt (siehe Anhang: Seite \pageref{ConfigurationHandler} ab Zeile 79). Dies fördert die Kapselung von Daten und bietet eine klare 
Schnittstelle durch eine Hauptmethode, die interne Methoden aufruft und so die Funktionalität zentral verwaltet.