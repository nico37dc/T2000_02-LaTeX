\section{Methoden der Testkonfiguration}

Zur Konfiguration von Tests stehen verschiedene Methoden zur Verfügung. Diese Methoden werden im Folgenden kurz dargestellt, 
um eine fundierte Entscheidung über die am besten geeignete Konfigurationsmethode treffen zu können.

Die erste Methode ist die direkte Konfiguration der Testparameter im Quellcode. Dieser Ansatz kann in C++ durch die Definition von Konstanten, statischen Variablen 
oder speziellen Konfigurationsklassen realisiert werden. Dies bietet eine genaue Kontrolle über die Testausführung, schränkt jedoch die Flexibilität ein, da Änderungen 
an der Konfiguration eine Neukompilierung des Codes erfordern.

Es gibt auch die Möglichkeit, externe Konfigurationsdateien, wie JSON- oder XML-Dateien, zu verwenden. Diese Methode bietet Flexibilität, da sie die Verwaltung von Testparametern ermöglicht, 
ohne dass der Code neu kompiliert werden muss. Es ist jedoch erforderlich, dass die Konfigurationsdateien korrekt in den Quellcode eingebunden und verarbeitet werden. Zusätzlich
ist die Auswahl des Dateiformats entscheidend für die Umsetzung.

Eine weitere Möglichkeit ist die Verwendung von Terminaleingaben. Dies ermöglicht eine schnelle Anpassung der Testparameter, ohne den 
Quellcode oder externe Dateien ändern zu müssen. Ein Nachteil dieser Methode ist jedoch, dass bei einer großen Anzahl von Parametern die Handhabung umständlich wird.

Eine grafische Benutzeroberfläche (\acs{GUI}) ermöglicht es auch technisch weniger versierten Benutzern die Testparameter einfach und intuitiv zu konfigurieren. 
Die \acs{GUI} kann so gestaltet werden, dass sie eine visuelle Darstellung der verschiedenen Prüfparameter und Konfigurationsoptionen bietet. Die Entwicklung einer solchen 
Benutzeroberfläche erfordert jedoch zusätzliche Ressourcen und technisches Know-how in der \acs{GUI}-Entwicklung. Zudem ist die Wartung und Aktualisierung aufwändig.

Die Entwicklung eines Testprogramms erfordert eine sorgfältige Planung der Testkonfiguration, um sicherzustellen, dass die Tests sowohl präzise als auch 
anpassungsfähig sind. Die Wahl der geeigneten Tools und Techniken hängt stark von den spezifischen Anforderungen des zu testenden Systems und der vorhandenen Infrastruktur ab.

\citep{Witte2019} \citep{Beneken2022}