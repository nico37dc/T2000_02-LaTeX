%!TEX root = ../dokumentation.tex

%
% Nahezu alle Einstellungen koennen hier getaetigt werden
%

% Zeichencodierung
\usepackage[utf8]{inputenc}
\usepackage[T1]{fontenc}
\usepackage{lmodern}
\usepackage[official]{eurosym}

% Zeilenabstand
\usepackage[onehalfspacing]{setspace}


% Seitenränder anpassen 
% \usepackage[paper=a4paper,left=2.5cm,right=2.5cm,top=2.5cm,bottom=2.5cm]{geometry}
\usepackage{geometry}
\setlength{\topskip}{\ht\strutbox} % behebt Warnung von geometry 
\setlength{\headheight}{1.1\baselineskip}   % Headheight hochsetzten 
\renewcommand*{\chapterheadstartvskip}{\vspace*{.5\baselineskip}}% Abstand einstellen

% PDF als ganze Seite einbinden
\usepackage{pdfpages}

% Seitengroesse
\usepackage{fullpage}

% Zeilenumbruch und mehr
\usepackage[activate]{microtype}

% gestrichelte Linien
\usepackage{dashrule}

% Index-Erstellung
\usepackage{makeidx}

% Lokalisierung (neue deutsche Rechtschreibung)
\usepackage[ngerman]{babel}

% Anführungszeichen 
\usepackage[babel,german=quotes]{csquotes}

% Grafiken
\usepackage{graphicx}
\usepackage{subfig}
\usepackage{caption}
\DeclareCaptionLabelFormat{blank}{}
% \usepackage{subcaption}
\usepackage{float}
\restylefloat{figure}

% Mathematische Textsaetze
\usepackage{amsmath}
\usepackage{amssymb}

% checkmarks
\usepackage{pifont}
\newcommand{\cmark}{\ding{51}}%
\newcommand{\xmark}{\ding{55}}%
% footnotes in a table
\usepackage{setspace}
\usepackage{threeparttable}

% Farben
\usepackage{color}
\definecolor{LinkColor}{rgb}{0,0,0.2}
\definecolor{ListingBackground}{rgb}{0.92,0.92,0.92}

% PDF Einstellungen
\usepackage{hyperref}
[%
	%pdftitle={\pdftitel},
	%pdfauthor={\autor},
	%pdfsubject={\arbeit},
	pdfcreator={pdflatex, LaTeX with KOMA-Script},
	pdfpagemode=UseOutlines, % Beim Oeffnen Inhaltsverzeichnis anzeigen
	pdfdisplaydoctitle=true, % Dokumenttitel statt Dateiname anzeigen.
	pdflang=de % Sprache des Dokuments.
]

% (Farb-)einstellungen für die Links im PDF
\hypersetup{%
	colorlinks=true, % Aktivieren von farbigen Links im Dokument
	linkcolor=black, % Farbe festlegen
	citecolor=black,
	filecolor=LinkColor,
	menucolor=LinkColor,
	urlcolor=black,  %%changed to black	
	bookmarksnumbered=true % Überschriftsnummerierung im PDF Inhalt anzeigen.
}

% Literaturverweise 
\usepackage[authoryear]{natbib}
\newcommand{\literaturverz}[1]{
	%Autorenverknüpfung mit "und"
	\renewcommand{\harvardand}{and}
	\bibliography{#1}
}

% http://projekte.dante.de/DanteFAQ/Silbentrennung
\usepackage{microtype}
\clubpenalty=10000
\widowpenalty=10000
\displaywidowpenalty=10000

% Quellcode
% Einbinden des Listings-Pakets für Code-Listings (welches??)

%
%
%
%

% Glossar
\usepackage[
	nonumberlist, %keine Seitenzahlen anzeigen
	acronym,      %ein Abkürzungsverzeichnis erstellen
	%section,     %im Inhaltsverzeichnis auf section-Ebene erscheinen
	toc,          %Einträge im Inhaltsverzeichnis
]{glossaries}

% Abkürzungsverzeichnis
\usepackage[nohyperlinks]{acronym}

% Bildpfad
\graphicspath{{images/}}

% nur ein latex-Durchlauf für die Aktualisierung von Verzeichnissen nötig
\usepackage{bookmark}

% eine Kommentarumgebung "k" (Handhabe mit \begin{k}<Kommentartext>\end{k},
% Kommentare werden rot gedruckt). Wird \% vor excludecomment{k} entfernt,
% werden keine Kommentare mehr gedruckt.
\usepackage{comment}
\specialcomment{k}{\begingroup\color{red}}{\endgroup}
% \excludecomment{k}

% Color in tables
\usepackage{colortbl}