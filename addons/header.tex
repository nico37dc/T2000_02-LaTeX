%!TEX root = ../dokumentation.tex

%
% Nahezu alle Einstellungen koennen hier getaetigt werden
%

% Zeichencodierung
\usepackage[utf8]{inputenc}
\usepackage[T1]{fontenc}
\usepackage{lmodern}
\usepackage[official]{eurosym}

% Zeilenabstand
\usepackage[onehalfspacing]{setspace}


% Seitenränder anpassen 
%\usepackage[paper=a4paper,left=2.5cm,right=2.5cm,top=2.5cm,bottom=2.5cm]{geometry}
\usepackage{geometry}
\setlength{\topskip}{\ht\strutbox} % behebt Warnung von geometry 
\setlength{\headheight}{1.1\baselineskip}   % Headheight hochsetzten 
\renewcommand*{\chapterheadstartvskip}{\vspace*{.5\baselineskip}}% Abstand einstellen

%PDF als ganze Seite einbinden
\usepackage{pdfpages}

%Seitengroesse
\usepackage{fullpage}

%Zeilenumbruch und mehr
\usepackage[activate]{microtype}

% gestrichelte Linien
\usepackage{dashrule}

% Index-Erstellung
\usepackage{makeidx}

% Lokalisierung (neue deutsche Rechtschreibung)
\usepackage[ngerman]{babel}

% Anführungszeichen 
\usepackage[babel,german=quotes]{csquotes}

% Grafiken
\usepackage{graphicx}
\usepackage{subfig}
\usepackage{caption}
\DeclareCaptionLabelFormat{blank}{}
%\usepackage{subcaption}
\usepackage{float}
\restylefloat{figure}

% Mathematische Textsaetze
\usepackage{amsmath}
\usepackage{amssymb}

%checkmarks
\usepackage{pifont}
\newcommand{\cmark}{\ding{51}}%
\newcommand{\xmark}{\ding{55}}%
%footnotes in a table
\usepackage{setspace}
\usepackage{threeparttable}

% Farben
\usepackage{color}
\definecolor{LinkColor}{rgb}{0,0,0.2}
\definecolor{ListingBackground}{rgb}{0.92,0.92,0.92}

% PDF Einstellungen
\usepackage{hyperref}
[%
	%pdftitle={\pdftitel},
	%pdfauthor={\autor},
	%pdfsubject={\arbeit},
	pdfcreator={pdflatex, LaTeX with KOMA-Script},
	pdfpagemode=UseOutlines, % Beim Oeffnen Inhaltsverzeichnis anzeigen
	pdfdisplaydoctitle=true, % Dokumenttitel statt Dateiname anzeigen.
	pdflang=de % Sprache des Dokuments.
]

% (Farb-)einstellungen für die Links im PDF
\hypersetup{%
	colorlinks=true, % Aktivieren von farbigen Links im Dokument
	linkcolor=black, % Farbe festlegen
	citecolor=black,
	filecolor=LinkColor,
	menucolor=LinkColor,
	urlcolor=black,  %%changed to black	
	bookmarksnumbered=true % Überschriftsnummerierung im PDF Inhalt anzeigen.
}

% Literaturverweise 
\usepackage[authoryear]{natbib}
\newcommand{\literaturverz}[1]{
	%Autorenverknüpfung mit "und"
	\renewcommand{\harvardand}{and}
	\bibliography{#1}
}

% http://projekte.dante.de/DanteFAQ/Silbentrennung
\usepackage{microtype}
\clubpenalty=10000
\widowpenalty=10000
\displaywidowpenalty=10000

% Quellcode

\usepackage{listings} % Einbinden des Listings-Pakets für Code-Listings

% Definition von benutzerdefinierten Farben
\definecolor{KeywordPurple}{rgb}{0.5, 0.0, 0.5}
\definecolor{CommentGreen}{rgb}{0.13, 0.55, 0.13}
\definecolor{StringOrange}{rgb}{0.8, 0.33, 0.0}
\definecolor{TypeBlue}{rgb}{0.2, 0.2, 0.8} % Sanfteres Blau
\definecolor{IntMagenta}{rgb}{0.8, 0.2, 0.8} % Sanfteres Magenta
\definecolor{ClassGreen}{rgb}{0.0, 0.75, 0.5} % Türkis, weniger grell
\definecolor{FunctionsYellow}{rgb}{0.9, 0.8, 0.1} % Weniger grelles Gelb
% Definition eines benutzerdefinierten JSON-Sprachstils
\lstdefinelanguage{json}{
    morestring=[b]",                % Definiert das Anführungszeichen als String-Begrenzer
    stringstyle=\color{CommentGreen},    % Setzt die Farbe der Strings auf Grün
    literate=                       % Ermöglicht die Formatierung einzelner Zeichen oder Zeichenfolgen
     *{0}{{{\color{IntMagenta}0}}}{1}
      {1}{{{\color{IntMagenta}1}}}{1}
      {2}{{{\color{IntMagenta}2}}}{1}
      {3}{{{\color{IntMagenta}3}}}{1}
      {4}{{{\color{IntMagenta}4}}}{1}
      {5}{{{\color{IntMagenta}5}}}{1}
      {6}{{{\color{IntMagenta}6}}}{1}
      {7}{{{\color{IntMagenta}7}}}{1}
      {8}{{{\color{IntMagenta}8}}}{1}
      {9}{{{\color{IntMagenta}9}}}{1}
      {"Test}{{{\color{StringOrange}"Test}}}{4}
      {ID"}{{{\color{StringOrange}ID"}}}{2}
      {Description"}{{{\color{StringOrange}Description"}}}{10}
      {"..//"}{{{\color{StringOrange}"..//"}}}{4},
}

% Definition eines benutzerdefinierten JSON-Stils
\lstdefinestyle{myJSON}{
	backgroundcolor=\color{white},   % Setzt die Hintergrundfarbe auf Weiß
	basicstyle=\scriptsize\ttfamily, % Setzt die Basis-Schriftart und -größe
	breaklines=true,                 % Ermöglicht das Umbruch von langen Zeilen
	captionpos=b,                    % Position der Beschriftung unterhalb des Listings
	frame=single,                    % Rahmenart: einfacher Rahmen um das Listing
	keepspaces=true,                 % Beibehalten von Leerzeichen im Code, nützlich für die Einrückung
	language=json,                   % Setzt die Sprache des Listings auf JSON
	numbers=left,                    % Zeilennummern auf der linken Seite
	numbersep=5pt,                   % Abstand der Zeilennummern zum Text
	numberstyle=\scriptsize,         % Stil und Größe der Zeilennummern
	rulecolor=\color{black},         % Farbe des Rahmens
	showstringspaces=false,          % Keine speziellen Markierungen für Leerzeichen in Strings
	tabsize=2,                       % Größe eines Tabulators ist 2 Leerzeichen
	keywordstyle=[2]\color{TypeBlue},    % Farbe der Schlüsselwörter (true, false) auf Blau setzen
	keywords=[2]{true, false}        % Definiert zusätzliche Schlüsselwörter
}

\lstloadlanguages{C++}

% Definition eines benutzerdefinierten C++-Stils
\lstdefinestyle{myCPP}{
	backgroundcolor=\color{white},
	basicstyle=\scriptsize\ttfamily,
	breaklines=true,
	breakindent=20pt,
	% breakatwhitespace=true, 
	postbreak=\mbox{\textcolor{gray}{$\hookrightarrow$}\space},
	captionpos=b,
	frame=single,
	keepspaces=true,
	language=C++,
	numbers=left,
	numbersep=5pt,
	numberstyle=\scriptsize,
	rulecolor=\color{black},
	showstringspaces=false,
	tabsize=2,
	stringstyle=\color{StringOrange},
	commentstyle=\color{CommentGreen},
	morekeywords={string, time_t, size_t, curl_easy_setopt},
	keywordstyle=\color{TypeBlue},
	emph={using, return, if, else, switch, case, break},
	emphstyle=\color{KeywordPurple},
	emph={[2]std, chrono, stringstream, system_clock, helper, RequestProvider, shared_ptr, mvIMPACT, acquire, ImageBuffer, this_thread, filesystem, milliseconds, ImpactAcquireException, spdlog, ifstream, nlohmann, level, DeviceManager, termios, json, CURL, CURLcode, ConfigurationParameter, ThreadParameter, Device, exception, Statistics, TimestampProvider, ConfigurationHandler, ImageSavePreparer, Logger, AzureIssueCreator},
	emphstyle={[2]\color{ClassGreen}},
	emph={[3]getTimestamp, str, localtime, put_time, processRequest, isOK, duration_cast, getDeviceFromUserInput, sleep_for, acquisitionStop, open, parse, getBufferPart, getErrorCodeAsString, acquisitionStart, reset, ref, getImageBufferDesc, getBuffer, save, now, basic_logger_mt, curl_slist_free_all, curl_easy_cleanup, tcgetattr, fcntl, getchar, ungetc, base64_encode, curl_easy_init, curl_slist_append, curl_easy_perform, curl_easy_strerror, create_directories, ceil, name, readS, read, append, to_string, mkdir, set_level, set_pattern, rotating_logger_mt, flush_on, c_str, time_since_epoch, setfill, setw, is_string, is_boolean, is_number_integer, count, get, empty, isalnum, to_time_t, getTimestamp_ms, getNowTime, operator=, loadConfiguration, checkPath, createFolder, readConfigurationParameter, checkConfigurationParameter, createOutputDirectories, prepareImageSave, prepareBufferSave, createImageSubpathFolder, createBufferSubpathFolder, initializeLogger, logStatistics, setLogLevel, interpretStatistics, sendAzureIssue, checkKeyboardHit, createAzureIssue, handleCurlResponse},
	emphstyle={[3]\color{FunctionsYellow}},
	escapeinside={@@},
	literate=
		{0}{{\textcolor{IntMagenta}{0}}}1
		{1}{{\textcolor{IntMagenta}{1}}}1
		{2}{{\textcolor{IntMagenta}{2}}}1
		{3}{{\textcolor{IntMagenta}{3}}}1
		{4}{{\textcolor{IntMagenta}{4}}}1
		{5}{{\textcolor{IntMagenta}{5}}}1
		{6}{{\textcolor{IntMagenta}{6}}}1
		{7}{{\textcolor{IntMagenta}{7}}}1
		{8}{{\textcolor{IntMagenta}{8}}}1
		{9}{{\textcolor{IntMagenta}{9}}}1
}


\lstdefinestyle{myHPP}{
	backgroundcolor=\color{white},
	basicstyle=\scriptsize\ttfamily,
	breaklines=true,
	breakindent=20pt,
	breakatwhitespace=true, 
	postbreak=\mbox{\textcolor{gray}{$\hookrightarrow$}\space},
	captionpos=b,
	frame=single,
	keepspaces=true,
	language=C++,
	numbers=left,
	numbersep=5pt,
	numberstyle=\scriptsize,
	rulecolor=\color{black},
	showstringspaces=false,
	tabsize=2,
	stringstyle=\color{StringOrange},
	moredelim=[s][\color{StringOrange}]{<}{>},
	commentstyle=\color{CommentGreen},
	morekeywords={string, time_t, size_t},
	keywordstyle=\color{TypeBlue},
	emph={using},
	emphstyle=\color{KeywordPurple},
	emph={[2]std, ConfigurationParameter, ThreadParameter, Device, Statistics, TimestampProvider, ConfigurationHandler, ImageSavePreparer, Logger, AzureIssueCreator},
	emphstyle={[2]\color{ClassGreen}},
	emph={[3]getTimestamp, getTimestamp_ms, getNowTime, operator=, loadConfiguration, checkPath, createFolder, readConfigurationParameter, checkConfigurationParameter, createOutputDirectories, prepareImageSave, prepareBufferSave, createImageSubpathFolder, createBufferSubpathFolder, initializeLogger, logStatistics, setLogLevel, interpretStatistics, sendAzureIssue, checkKeyboardHit, createAzureIssue, handleCurlResponse},
	emphstyle={[3]\color{FunctionsYellow}},
	escapeinside={@@},
	literate=
		{0}{{\textcolor{IntMagenta}{0}}}1
		{1}{{\textcolor{IntMagenta}{1}}}1
		{2}{{\textcolor{IntMagenta}{2}}}1
		{3}{{\textcolor{IntMagenta}{3}}}1
		{4}{{\textcolor{IntMagenta}{4}}}1
		{5}{{\textcolor{IntMagenta}{5}}}1
		{6}{{\textcolor{IntMagenta}{6}}}1
		{7}{{\textcolor{IntMagenta}{7}}}1
		{8}{{\textcolor{IntMagenta}{8}}}1
		{9}{{\textcolor{IntMagenta}{9}}}1
}

% Glossar
\usepackage[
	nonumberlist, %keine Seitenzahlen anzeigen
	acronym,      %ein Abkürzungsverzeichnis erstellen
	%section,     %im Inhaltsverzeichnis auf section-Ebene erscheinen
	toc,          %Einträge im Inhaltsverzeichnis
]{glossaries}

%Abkürzungsverzeichnis
\usepackage[nohyperlinks]{acronym}

%Bildpfad
\graphicspath{{images/}}

%nur ein latex-Durchlauf für die Aktualisierung von Verzeichnissen nötig
\usepackage{bookmark}

% eine Kommentarumgebung "k" (Handhabe mit \begin{k}<Kommentartext>\end{k},
% Kommentare werden rot gedruckt). Wird \% vor excludecomment{k} entfernt,
% werden keine Kommentare mehr gedruckt.
\usepackage{comment}
\specialcomment{k}{\begingroup\color{red}}{\endgroup}
%\excludecomment{k}

%Color in tables
\usepackage{colortbl}
